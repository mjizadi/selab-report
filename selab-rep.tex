\documentclass[a4paper,notitlepage]{report}
\usepackage{xepersian}

\settextfont{XB Niloofar}
\setdigitfont[Scale=1]{XB Niloofar}
\setlatintextfont[Scale=1]{DejaVu Serif}
\title{درس‌نامهٔ آزمایشگاه مهندسی نرم‌افزار}
%\subtitle{دانشکدهٔ مهندسی برق و کامپیوتر دانشگاه تهران}
\author{
رامتین خسروی\\\small{استادیار دانشکدهٔ مهندسی برق و کامپیوتر دانشگاه تهران}\\
\small\texttt{<r.khosravi@ut.ac.ir>}\and
سید حامد حسینی‌نژاد\\\small{مدیر واحد آموزش، شرکت داتین}\\
\small\texttt{<h.hosseininejad@dotin.ir>}\and
محمدجواد ایزدی نجف‌آبادی\\\small{مدیر واحد کنترل کیفیت، شرکت داتین}\\
\small\texttt{<m.j.izadi@dotin.ir>}
}
\date{نیمسال دوم سال تحصیلی ۹۳-۹۴}
\begin{document}
\maketitle
\tableofcontents
\newpage

\begin{abstract}
در نیمسال دوم سال تحصیلی ۹۳-۹۴، برنامهٔ متفاوتی برای درس آزمایشگاه مهندسی نرم‌افزار دانشگاه تهران طراحی و اجرا شد
که در این درس‌نامه شرح آن ارائه خواهد شد.
برنامهٔ کلی عبارت بود از تعریف یک پروژه که در طول ترم توسط گروه‌های دانشجویان به طور کامل پیاده‌سازی شود.
روال اجرایی بر اساس متدولوژی‌های چابک و به طور خاص فرایند توسعه نرم‌افزار اسکرام طراحی شد.
در این فرایند دانشجویان به عنوان اعضای تیم تولید و دستیاران آموزشی به عنوان صاحبان محصول در نظر گرفته‌شدند.
همچنین نقش اسکرام‌مستر به صورت چرخشی توسط اعضای تیم به عهده گرفته‌شد.

نحوهٔ اجرای صحیح متدولوژی اسکرام توسط دسیاران آموزشی بررسی و ارزیابی شد،
به این صورت که دستیاران آموزشی علاوه بر حضور در جلسات به عنوان صاحب محصول، فعالیت‌های مختلف اعضای تیم را پایش و ارزیابی می‌کردند.
به عنوان مثال نحوهٔ ایفای صحیح نقش اسکرام‌مستر در جلسات مورد ارزیابی قرار می‌گرفت.
همچنین با ایجاد دسترسی‌های مناسب به سامانه‌های زیرساختی، دستیاران آموزشی میزان مشارکت دانشجویان در اجرای صحیح فرایند را نیز بررسی می‌کردند.

در کنار فرایند مذکور جلسات آموزشی در طول ترم جهت آشنا شدن دانشجویان با مسائل مهم مطرح در پروژه‌های نرم‌افزاری صنعتی برگزار شد.

در نهایت، ارزیابی‌های مستمر صورت‌گرفته در طول اجرای پروژه به همراه برگزاری جلسهٔ دموی نهایی نتیجهٔ ارزشیابی پایانی را مشخص نمود.
\end{abstract}
\chapter{مقدمه}
درس آزمایشگاه مهندسی نرم‌افزار به عنوان محلی برای تجربهٔ عملی مباحث آموخته‌شده در دروس نظری مهندسی نرم‌افزار در چارت درسی دانشجویان این رشته قرار گرفته‌است.
اما قرار گرفتن این نکته که بسیاری از مفاهیم مذکور بیشتر در چارچوب پروژه‌های بزرگ صنعتی ارزش واقعی خود را نشان می‌دهند، 
در کنار محدودیت‌های زمانی و زیرساختی موجود در محیط دانشگاه، طراحی این آزمایشگاه را دچار چالش می‌کند.

برای مواجهه با این مشکل، تصمیم گرفته شد یک پروژه به عنوان کار عملی دانشجویان این درس تعریف شود که از لحاظ محتوا بسیار ساده باشد،
اما برای اینکه  دانشجویان حضور در یک پروژهٔ صنعتی را تا حدی تجربه کنند، سعی بر این شود که نحوهٔ تعامل بین ذینفعان پروژه
با آنچه در دنیای مدرن صنعت نرم‌افزار در حال انجام است مشابهت داشته‌باشد.


در این راستا و در ادامهٔ همکاری‌های مابین دانشگاه تهران و شرکت نرم‌افزاری داتین، طرح پیش رو توسط کمیتهٔ شکل‌گرفته برای ارائهٔ این درس،
متشکل از استاد دروس مهندسی نرم‌افزار در دانشگاه تهران، و نمایندگان شرکت نرم‌افزاری داتین، تدوین گشت.
بدیهی است اجرای این طرح، علاوه بر  انتقال تجربیات صنعت‌گران به دانشجویان، می‌تواند مفاهیم و روش‌های نوین مطرح‌شده در پژوهش‌های مرتبط با نرم‌افزار را نیز
به صنعت در حال رشد نرم‌افزار کشور منتقل نماید.
\section{اهداف}
\chapter{توسعهٔ محصول}
همان گونه که پیش‌تر اشاره شد، طرح درس آزمایشگاه مهندسی نرم‌افزار شامل اجرای یک پروژه از ابتدا در انتها با رعایت استانداردهای روز صنعت نرم‌افزار خواهد بود.
در این راستا و با توجه به محدودیت‌های زمانی موجود، یک پروژهٔ نرم‌افزاری نسبتاً ساده برای اجرا توسط دانشجویان تعریف شد.
در کنار پیاده‌سازی نیازمندی‌های مطرح‌شده در صورت پروژه، لازم است دانشجویان یک فرایند مشخص تولید نرم‌افزار (مبتنی بر متدولوژی اسکرام) را نیز دنبال نمایند.
در هنگام تحویل و گزارش وضعیت اجرای پروژه نیز علاوه بر بررسی نیازمندی‌ها، رعایت برخی اصول مربوط با تولید نرم‌افزار نیز توسط دستیاران آموزشی کنترل و ارزیابی خواهد شد.
\section{تعریف پروژه}
پروژهٔ تعریف‌شده برای پیاده‌سازی توسط دانشجویان یک وب‌سایت اینترنتی برای میزبانی بازی «اسم و فامیل» است.
به طور کلی لازم است کاربران بتوانند پس از ثبت نام در این وب‌سایت و دریافت اطلاعات حساب کاربری، با ورود به سایت
نسبت به ساخت اتاق بازی جدید و یا پیوستن به اتاق‌های بازی موجود که توسط سایر کاربران ساخته‌شده‌است اقدام نمایند.
پس از ورود کاربران به اتاق بازی، سازندهٔ اتاق می‌تواند بازی را شروع کند.

شروع بازی شامل انتخاب تصادفی یک حرف برای بازی شرکت‌کنندگان است.
شرکت‌کنندگان کلمات مورد نظر را در فیلدهای متناظر وارد می‌کنند و پس از گذشت زمان مشخص و یا با اعلام دو نفر از شرکت‌کنندگان بازی پایان می‌یابد.
پس از اتمام بازی شرکت‌کنندگان کلمات نوشته‌شده توسط سایر شرکت‌کنندگان را ارزیابی نموده و سامانه به طور خودکار نتایج نهایی و امتیازات هر یک از بازیکنان را نمایش می‌دهد.

در یک اتاق بازی، ممکن است چند بازی به صورت پشت سرهم برگزار شوند.
پس از هر بازی، رتبه‌بندی بازیکنان تا پایان آخرین بازی نمایش داده می‌شود.
همچنین فهرست کلی امتیازات همهٔ کاربران وب‌سایت از صفحهٔ اصلی سامانه قابل دسترسی خواهد بود.
\section{فرایند توسعهٔ محصول}
با توجه به اقبال فراوانی که سازمان‌ها و شرکت‌های داخلی و خارجی به فرایند توسعهٔ محصول اسکرام نشان داده‌اند،
این فرایند به عنوان استاندارد روز صنعت نرم‌افزار در پروژه‌های با ماهیت چابک، جهت اجرا توسط دانشجویان انتخاب شد.
\subsection{نقش‌ها}
اسکرام سه نقش را به عنوان اعضای حاضر در این فرایند معرفی می‌کند:
\subsubsection{تیم}
در اسکرام، به جای توجه به اعضای تیم به عنوان افراد مجزا، به کل آن‌ها به عنوان «تیم» توجه می‌شود.
اسکرام در خصوص نحوهٔ اجرای کارها در داخل تیم و عناوین و مسؤولیت‌هایی که هر یک از اعضا می‌تواند داشته‌باشد صحبت چندانی نمی‌کند
و لازم می‌داند که تیم کاملاً مستقل و توسط خود اعضا مدیریت شود.
در مقابل اسکرام معتقد است که تیم باید توانایی انجام انواع مختلف کارهایی که در پیاده‌سازی محصول مورد نظر است را داشته‌باشد.
یعنی یک تیم توسعهٔ نرم‌افزار لازم است توانایی تحلیل، طراحی، پیاده‌سازی، تست و \ldots را داشته‌باشد.
اینکه این مسؤولیت‌ها هر کدام توسط یک شخص صورت می‌پذیرد یا همهٔ اعضای تیم در همهٔ آنها دخیل هستند از حیطهٔ اظهار نظر اسکرام خارج است.

در این پروژه با توجه به اینکه توسعهٔ محصول به عهدهٔ دانشجویان درس گذاشته‌شده، نقش «تیم» در واقع توسط ایشان ایفا می‌شود.
با توجه به ملاحظاتی که در ادامه توضیح داده خواهد شد، دانشجویان به تیم‌های دقیقاً ۴نفری تقسیم شدند.
این تقسیم‌بندی در درجهٔ اول با انتخاب خود دانشجویان و پس از آن در نظر گرفتن محدودیت مشخص‌شده در خصوص تعداد اعضای تیم اعمال شده‌است.

هر یک از این گروه‌های ۴نفری یک تیم اسکرام را تشکیل می‌دهند که محصول مورد نظر مشتری را به صورت مستقل و بر اساس نیازمندی‌هایی که توسط مالک محصول
به ایشان منتقل می‌شود پیاده‌سازی می‌کنند.
\subsubsection{مالک محصول}
طبق تعریف اسکرام، مالک محصول نمایندهٔ مشتری بوده و شخصی است که نیازمندی‌ها را از مشتری به تیم منتقل می‌کند.
مالک محصول مسؤول نگهداری بک‌لاگ محصول است که فهرستی از همهٔ نیازمندی‌های مطرح‌شده از سوی مشتری است.
این نیازمندی‌ها معمولاً به صورت روایت‌های کاربری، توسط مالک محصول نگاشته‌شده و بر اساس اولویت پیاده‌سازی در بک‌لاگ محصول مرتب‌سازی می‌شوند.
مالک محصول می‌تواند بر اساس زمان‌بندی ارائه‌شده توسط اعضای تیم، اولویت‌ها را تغییر داده و یا روایات را به روایات کوچک‌تر تقسیم نماید.

با توجه به اینکه صورت مسأله توسط دستیاران آموزشی مشخص شده و به دانشجویان منتقل می‌شود،
واضح است که این نقش باید توسط دستیاران آموزشی اجرا شود.
با توجه به تعداد گروه‌های دانشجویان و همچنین تعداد دستیاران آموزشی، هر گروه به یکی از دستیاران آموزشی محول می‌گردد.

لازم به ذکر است که هم در ابتدای شروع پروزه و هم در طول فرایند، به صورت مداوم این نکته به دانشجویان و اعضای تیم گوشزد می‌شود
که همهٔ فعالیت‌هایی که توسط دستیاران آموزشی انجام می‌گیرد، لزوماً با عنوان مالک محصول نبوده و عموماً با توجه به نیاز به ارزشیابی دانشجویان لازم است
دستیار آموزشی فعالیت‌هایی را خارج از کسوت مالکیت محصول انجام دهد.
\subsubsection{اسکرام‌مستر}
اسکرام‌مستر مسؤول اجرای صحیح فرایند اسکرام در تیم است.
شخصی که این مسؤولیت را به عهده می‌گیرد باید با استفاده از دانشی که در خصوص اهداف و استانداردهای اسکرام دارد، تیم را در راستای اجرای صحیح آن هدایت نماید.

با توجه به اینکه دانش اسکرام و اجرای صحیح فعالیت‌های آن از مباحث مهم دروس مهندسی نرم‌افزار بوده و در شرکت‌ها و سازمان‌ها نیز مورد اهمیت است،
تصمیم گرفته‌شد که این نقش به دانشجویان درس محول شود.
به این ترتیب، اعضای هر گروه به صورت گردشی این مسؤولیت را به عهده می‌گیرند و شخصی که در هر یک از اسپرینت‌ها این مسؤولیت را به عهده گرفته،
علاوه بر وظایفی که به عنوان عضو تیم دارد، لازم است به عنوان اسکرام‌مستر، بر اجرای صحیح اسکرام در تیم نظارت و کنترل داشته‌باشد.
\subsection{برنامهٔ زمانی}
مطابق استانداردهای تجربی اسکرام، برای یک پروزه با ابعاد پروژهٔ این درس، اسپرینت‌هایی با طول ۲ هفته مناسب هستند.
با توجه به اینکه ۱۶ هفته در یک نیمسال تحصیلی وجود دارد، در مجموع ۸ اسپرینت برای هر تیم قابل برنامه‌ریزی است.
به این ترتیب بک‌لاگ محصول به مرور و بر اساس سرعت تیم در پیاده‌سازی روایات کاربری، توسط مالک محصول به تیم منتقل می‌شود.
در هر اسپرینت که ۲ هفته به طول می‌انجامد، تیم پیاده‌سازی کامل تعدادی از روایات کاربری را به عهده می‌گیرد.
این چرخه تا پایان نیمسال تحصیلی ادامه پیدا می‌کند تا در انتها یک محصول کامل، مطابق با نیازمندی‌های مطرح‌شده و با کیفیت مورد انتظار دستیاران آموزشی و استاد درس، آماده شود.

با توجه به اینکه ۸ اسپرینت برای هر تیم در نظر گرفته‌شده، تصمیم گرفته‌شد که اعضای هر تیم ۴ نفر باشند تا هر یک از اعضا در ۲ اسپرینت نقش اسکرام‌مستر را نیز ایفا نماید.
به این ترتیب پس از هر بازهٔ مشتمل بر ۲ اسپرینت، اسکرام‌مستر بعدی نیز به صورت تصادفی توسط دستیار آموزشی انتخاب می‌شود.
\chapter{محتوای آموزشی}
\section{برنامه‌نویسی حرفه‌ای}
\section{مفاهیم، ابزارها و تکنولوژی‌های روز}
\section{شیوه‌های ارائهٔ مطالب فنی}
\chapter{ارزشیابی دانشجویان}
\section{ارزشیابی فردی}
\section{ارزشیابی تیمی}
\chapter{ارزیابی طرح}
\section{ارزیابی کیفیت اجرا} 
\section{سنجش میزان اثرگذاری}
\section{دست‌آوردها}
\chapter{خاتمه}
\section{راه پیش رو}
\section{تشکر و قدردانی}
\appendix
\chapter{آشنایی با متدولوژی سکرام}
\chapter{مفاهیم، تکنولوژی‌های و ابزارهای آموزش‌داده‌شده}
\chapter{تعریف جامع پروژه}
\end{document}
