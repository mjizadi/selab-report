\documentclass[a4paper,notitlepage]{report}
\usepackage{xepersian}

\settextfont{XB Niloofar}
\setdigitfont[Scale=1]{XB Niloofar}
\setlatintextfont[Scale=1]{DejaVu Serif}
\title{درس‌نامهٔ آزمایشگاه مهندسی نرم‌افزار}
%\subtitle{دانشکدهٔ مهندسی برق و کامپیوتر دانشگاه تهران}
\author{
رامتین خسروی\\\small{استادیار دانشکدهٔ مهندسی برق و کامپیوتر دانشگاه تهران}\\
\small\texttt{<r.khosravi@ut.ac.ir>}\and
سید حامد حسینی‌نژاد\\\small{مدیر واحد آموزش، شرکت داتین}\\
\small\texttt{<h.hosseininejad@dotin.ir>}\and
محمدجواد ایزدی نجف‌آبادی\\\small{مدیر واحد کنترل کیفیت، شرکت داتین}\\
\small\texttt{<m.j.izadi@dotin.ir>}
}
\date{نیمسال دوم سال تحصیلی ۹۳-۹۴}
\begin{document}
\maketitle
\tableofcontents
\begin{abstract}
در نیمسال دوم سال تحصیلی ۹۳-۹۴، برنامهٔ متفاوتی برای درس آزمایشگاه مهندسی نرم‌افزار دانشگاه تهران طراحی و اجرا شد
که در این درس‌نامه شرح آن ارائه خواهد شد.
برنامهٔ کلی عبارت بود از تعریف یک پروژه که در طول ترم توسط گروه‌های دانشجویان به طور کامل پیاده‌سازی شود.
روال اجرایی بر اساس متدولوژی‌های چابک و به طور خاص فرایند توسعه نرم‌افزار اسکرام طراحی شد.
در این فرایند دانشجویان به عنوان اعضای تیم تولید و دستیاران آموزشی به عنوان صاحبان محصول در نظر گرفته‌شدند.
همچنین نقش اسکرام‌مستر به صورت چرخشی توسط اعضای تیم به عهده گرفته‌شد.

نحوهٔ اجرای صحیح متدولوژی اسکرام توسط دسیاران آموزشی بررسی و ارزیابی شد،
به این صورت که دستیاران آموزشی علاوه بر حضور در جلسات به عنوان صاحب محصول، فعالیت‌های مختلف اعضای تیم را پایش و ارزیابی می‌کردند.
به عنوان مثال نحوهٔ ایفای صحیح نقش اسکرام‌مستر در جلسات مورد ارزیابی قرار می‌گرفت.
همچنین با ایجاد دسترسی‌های مناسب به سامانه‌های زیرساختی، دستیاران آموزشی میزان مشارکت دانشجویان در اجرای صحیح فرایند را نیز بررسی می‌کردند.

در کنار فرایند مذکور جلسات آموزشی در طول ترم جهت آشنا شدن دانشجویان با مسائل مهم مطرح در پروژه‌های نرم‌افزاری صنعتی برگزار شد.

در نهایت، ارزیابی‌های مستمر صورت‌گرفته در طول اجرای پروژه به همراه برگزاری جلسهٔ دموی نهایی نتیجهٔ ارزشیابی پایانی را مشخص نمود.
\end{abstract}
\chapter{مقدمه}
درس آزمایشگاه مهندسی نرم‌افزار به عنوان محلی برای تجربهٔ عملی مباحث آموخته‌شده در دروس نظری مهندسی نرم‌افزار در چارت درسی دانشجویان این رشته قرار گرفته‌است.
اما قرار گرفتن این نکته که بسیاری از مفاهیم مذکور بیشتر در چارچوب پروژه‌های بزرگ صنعتی ارزش واقعی خود را نشان می‌دهند، 
در کنار محدودیت‌های زمانی و زیرساختی موجود در محیط دانشگاه، طراحی این آزمایشگاه را دچار چالش می‌کند.

برای مواجهه با این مشکل، تصمیم گرفته شد یک پروژه به عنوان کار عملی دانشجویان این درس تعریف شود که از لحاظ محتوا بسیار ساده باشد،
اما برای اینکه  دانشجویان حضور در یک پروژهٔ صنعتی را تا حدی تجربه کنند، سعی بر این شود که نحوهٔ تعامل بین ذینفعان پروژه
با آنچه در دنیای مدرن صنعت نرم‌افزار در حال انجام است مشابهت داشته‌باشد.


در این راستا و در ادامهٔ همکاری‌های مابین دانشگاه تهران و شرکت نرم‌افزاری داتین، طرح پیش رو توسط کمیتهٔ شکل‌گرفته برای ارائهٔ این درس،
متشکل از استاد دروس مهندسی نرم‌افزار در دانشگاه تهران، و نمایندگان شرکت نرم‌افزاری داتین، تدوین گشت.
بدیهی است اجرای این طرح، علاوه بر  انتقال تجربیات صنعت‌گران به دانشجویان، می‌تواند مفاهیم و روش‌های نوین مطرح‌شده در پژوهش‌های مرتبط با نرم‌افزار را نیز
به صنعت در حال رشد نرم‌افزار کشور منتقل نماید.
\section{اهداف}
\chapter{کار عملی}
همان گونه که پیش‌تر اشاره شد، طرح درس آزمایشگاه مهندسی نرم‌افزار شامل اجرای یک پروژه از ابتدا در انتها با رعایت استانداردهای روز صنعت نرم‌افزار خواهد بود.
در این راستا و با توجه به محدودیت‌های زمانی موجود، یک پروژهٔ نرم‌افزاری نسبتاً ساده برای اجرا توسط دانشجویان تعریف شد.
در کنار پیاده‌سازی نیازمندی‌های مطرح‌شده در صورت پروژه، لازم است دانشجویان یک فرایند مشخص تولید نرم‌افزار (مبتنی بر متدولوژی اسکرام) را نیز دنبال نمایند.
در هنگام تحویل و گزارش وضعیت اجرای پروژه نیز علاوه بر بررسی نیازمندی‌ها، رعایت برخی اصول مربوط با تولید نرم‌افزار نیز توسط دستیاران آموزشی کنترل و ارزیابی خواهد شد.
\section{تعریف پروژه}
پروژهٔ تعریف‌شده برای پیاده‌سازی توسط دانشجویان یک وب‌سایت اینترنتی برای میزبانی بازی «اسم و فامیل» است.
به طور کلی لازم است کاربران بتوانند پس از ثبت نام در این وب‌سایت و دریافت اطلاعات حساب کاربری، با ورود به سایت
نسبت به ساخت اتاق بازی جدید و یا پیوستن به اتاق‌های بازی موجود که توسط سایر کاربران ساخته‌شده‌است اقدام نمایند.
پس از ورود کاربران به اتاق بازی، سازندهٔ اتاق می‌تواند بازی را شروع کند.

شروع بازی شامل انتخاب تصادفی یک حرف برای بازی شرکت‌کنندگان است.
شرکت‌کنندگان کلمات مورد نظر را در فیلدهای متناظر وارد می‌کنند و پس از گذشت زمان مشخص و یا با اعلام دو نفر از شرکت‌کنندگان بازی پایان می‌یابد.
پس از اتمام بازی شرکت‌کنندگان کلمات نوشته‌شده توسط سایر شرکت‌کنندگان را ارزیابی نموده و سامانه به طور خودکار نتایج نهایی و امتیازات هر یک از بازیکنان را نمایش می‌دهد.

در یک اتاق بازی، ممکن است چند بازی به صورت پشت سرهم برگزار شوند.
پس از هر بازی، رتبه‌بندی بازیکنان تا پایان آخرین بازی نمایش داده می‌شود.
همچنین فهرست کلی امتیازات همهٔ کاربران وب‌سایت از صفحهٔ اصلی سامانه قابل دسترسی خواهد بود.
\section{فرایند توسعهٔ محصول}
\subsection{برنامهٔ زمانی}
\section{نقش‌ها}
\chapter{محتوای آموزشی}
\section{برنامه‌نویسی حرفه‌ای}
\section{مفاهیم، ابزارها و تکنولوژی‌های روز}
\section{شیوه‌های ارائهٔ مطالب فنی}
\chapter{ارزشیابی دانشجویان}
\section{ارزشیابی فردی}
\section{ارزشیابی تیمی}
\chapter{ارزیابی طرح}
\section{ارزیابی کیفیت اجرا} 
\section{سنجش میزان اثرگذاری}
\section{دست‌آوردها}
\chapter{خاتمه}
\section{راه پیش رو}
\section{تشکر و قدردانی}
\end{document}
